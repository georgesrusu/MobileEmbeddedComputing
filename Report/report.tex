\documentclass[a4paper,10pt]{article}
\usepackage[utf8]{inputenc}
\usepackage[francais]{babel}
\usepackage{indentfirst}
\usepackage{listings}
\usepackage{graphicx}
\usepackage{blindtext}
\usepackage{enumitem}
\usepackage{hyperref}
\usepackage[top=2.5cm,bottom=2.5cm,left=2.5cm,right=2.5cm]{geometry}
\pagestyle{headings}
\title{Group project}
\author{Ilias Boulif, Jérémy Bricq, George Rusu}
\date{\today}
\begin{document}
\maketitle
\tableofcontents
\newpage
\section{Introduction}

The aim of this project is to propose an implementation of an IoT network using \textit{Rime} where the sensor data is published through an MQTT-Rime gateway to normal MQTT subscribers. To achieve that we have implemented as requested by the project's statement, a tree-based routing protocol using Rime and an MQTT-Rime gateway.

Firstly we are going to explain the general structure of our system. Then we are going to present the network organization and routing of our network. And finally we are going to discuss about the optimizations that we have implemented.

The source code of this project implemenation is available at \cite{github}.


\section{General structure of our system}
%-le réseau basé sur un root , transmet les info au subsc
%-des sensors node qui recup les info et les envoi au root
%-les sensors se connecte entre par un réseau sans fil (rime)
%il comminuque entre en s'envoyant diff msg pour s'assurer du bon fct du reseau
%-le réseau est pensé pour pouvoir envoyé des info au root meme si il y a des sensors qui sont déplacé ou suprimé
%- le root est connecté au gatesay qui permet d'envoyer vers subscriber
%- le root recoit mode depuis le gateway , il est envoyé dans les alives responses packet jusqu'en bas

Our system is a network composed of one root node and a number of sensor nodes. The network depends on the root which is the node responsible for transmitting the information to the subscribers. The sensor nodes for their part, collect the information for the root and forward them to him.  using rime ! The nodes communicate by sending different kind of messages (see Section \ref{msgFormat}. Some of these messages are used to ensure a good working of the system. The network is built to keep sending information to the root even if there is a node is moved or deleted.  The root is connected to the gateway in order to send the information to the subscribers. Some modes are built in the system. These modes describe the system's behaviour. The root receives the mode through the gateway, and forward it the nodes by the mean of the alive response packets. 



\subsection{Message format}
\label{msgFormat}
In our implementation we have 3 packets structures. The node interprets the packet according to his type. We will explain each kind of packets and discuss the different types used by nodes to communicate between them.

\paragraph{Description of \textit{packet}}
\begin{enumerate}
\item Type : [DISCOVERY\_REQUEST, DISCOVERY\_RESPONSE, ALIVE\_REQUEST, ALIVE\_RESPONSE]
\item Rank : the rank of the emission node
\item Mode : the mode of sent [DATA\_ON\_CHANGE, DATA\_PERIODICALLY]
\item HaveSubscriber : Boolean that indicates if someone is subscribed to some topics
\end{enumerate}

\paragraph{Description of \textit{data\_packet} }
\begin{enumerate}
\item Type : [SENSOR\_DATA]
\item NodeSrc :  The node where the information come from
\item NodeRank : The rank of the node source
\item DataTemp : The temperature sent
\item DataOthers : A place used for another kind of data
\end{enumerate}

\paragraph{Description of \textit{data\_packet\_aggregate} }
\begin{enumerate}
\item Type : [SENSOR\_DATA\_AGGREGATE]
\item numberPacket : The number of packets
\item packet1 :  The first packet
\item packet2 : The second packet
\end{enumerate}

\paragraph{DISCOVERY\_REQUEST}  When a node starts, he is alone and has a rank of 0. To connect to a network, Each sensor node need to send in broadcast a discovery request to find a parent connected to that network. Even when connected, the node will still keep sending discovery request in order to find a better parent, if exists. The lower is the rank (except a rank of 0) and the nearest is the node from the root node.

\paragraph{DISCOVERY\_RESPONSE} When a node receives a discovery request packet he replies with a discovery response with his rank. The node that has made the request receive now the response and assign the his rank the value of the rank of his parent + 1.

\paragraph{ALIVE\_REQUEST} Each node has to keep sending an alive request in order to know if his parent is still available in the network. After 4 alive requests without any response, the child considers that he has no parent. If he has no more parent, his rank return back to 0 and he has to research for a new one. 

\paragraph{ALIVE\_RESPONSE } When a node receives an alive request, he has to replies with an alive response to his child to validate his presence. Each alive response is accompanied with the rank of the parent. In this way, if the parent is disconnected from the network (rank = 0), the child can also put his rank to 0 and can disconnect. 

\paragraph{SENSOR\_DATA } A sensor data packet is a packet that contains some information collected by the sensor. This packet travels from his sensor to the subscribers by going forwarded by the nodes on the path to the root.

\paragraph{SENSOR\_DATA\_AGGREGATE } A sensor data aggregate packet is a packet that contains many sensors data packets in one. This packet will be explained in details in the Section \ref{opti} : Optimizations.





\section{Network organization and routing}

\section{Optimizations}
\label{opti}

\section{Conclusion}

\begin{thebibliography}{9}

\bibitem{github}
Github repository,
\textit{MobileEmbeddedComputing}, Available at :
\url{https://github.com/georgesrusu/MobileEmbeddedComputing}.


\end{thebibliography}
\end{document}